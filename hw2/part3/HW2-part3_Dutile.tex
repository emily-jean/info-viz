\documentclass{neu_handout}
\usepackage{url}
\usepackage{amssymb}
\usepackage{amsmath}
\usepackage{marvosym}
\usepackage{graphicx}
\usepackage[pdftex]{graphicx}
\usepackage{subfigure}
\graphicspath{ {images/} }
\everymath{\displaystyle}

% Professor/Course information
\title{Homework 2 - Part 3}
\author{Emily Dutile}
\date{February 2018}
\course{CS7295}{Info Viz}

\begin{document}


\section*{Journal Paper Review - “Energy Portfolio Analysis”}

\subsubsection*{Step 1}
Matthew Brehmer, Jocelyn Ng, Kevin Tate, and Tamara Munzner discuss the overhaul of analysis software used for understanding energy use. The software helps anyone within the utilities and facilities management domain better understand how buildings consume energy. Within the paper, limitations around software that was previously developed is discussed, along with this software functions and implementations in order to meet the needs of energy workers.\\

The interactive GUI allows users to compare and filter different buildings, and a number of variables are factored into the visualizations such as usage statistics, categorical data (such as the type of building or number of occupants), and time series data. This article demonstrates a great framework for developing tools in order to better analyze energy usage, and it also shows that the sandbox created can better visualize complex, multi-dimensional databases.\\

The article provides a significant amount of evidence, making the argument and research valid. Several of the designs that were created in the sandbox environment were adopted in the newer version of the Energy Manager software. Before reading this paper, I did not know much about the software needs of energy workers and how greatly these visualizations can impact their decision process. For a relatively technical topic, the paper was not overly saturated with technical domain knowledge and was easy to comprehend. \\



\subsubsection*{Step 2}
While exploring the sandbox environment create by Dr. Brehmer as part of this paper’s
research, the tool seamlessly visualizes and interpreta a significant amount of meta-data with easy and speed. There are some minor visualize enhancements that I would make, such as fixing the overlapping labels or reducing the view so you don't need to scroll so much during an analysis.

\end{document}
