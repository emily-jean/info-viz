\documentclass{neu_handout}
\usepackage{url}
\usepackage{amssymb}
\usepackage{amsmath}
\usepackage{marvosym}
\usepackage{graphicx}
\usepackage[pdftex]{graphicx}
\usepackage{subfigure}
\graphicspath{ {images/} }
\everymath{\displaystyle}

% Professor/Course information
\title{Homework 2 - Part 1}
\author{Emily Dutile}
\date{February 2018}
\course{CS7295}{Info Viz}

\begin{document}


\section*{“Visualization Zoo}

A Tour through the Visualization Zoo\\
Jeffrey Heer, Michael Bostock, Vadim Ogievetsky\\
Communications of the ACM, 53(6), pp. 59-67, 2010\\
https://idl.cs.washington.edu/files/2012-VisualizationZoo-CACM.pdf\\

The article 'A Tour through the Visualization Zoo', written by Jeffrey Heer, Michael Bostock, and Vadim Ogievetsky, explores visualization methods in which different datasets can be displayed through multiple visual encodings. The authors write about exploring multiple ways of visualizing a dataset and how challenging it is for any given data set due to the number of visual encodings, and then selecting an appropriate encoding to map data values to graphical features (i.e.: color, position, shape, and size). The individuals describe a collection of 17 visualizations that are divided up into 5 categories by their data types. These categories consisted of Hierarchies, Maps, Networks, Time Series Data, and Statistical Distributions. The authors express in detail the importance of utilizing our ability to see patterns, trends, and outliers through visuals. Throughout the reading, the authors utilize the same dataset, making it very comprehensible and easy to see the complexities and the importance around what questions to ask in order to identify the appropriate data and then the encodings. The article introduced a few newer visualizations for me, such as Q-Q plots and a specific tree layout, the icicle tree.


\end{document}
