\documentclass{neu_handout}
\usepackage{url}
\usepackage{amssymb}
\usepackage{amsmath}
\usepackage{marvosym}
\usepackage{graphicx}
\usepackage[pdftex]{graphicx}
\usepackage{subfigure}
\graphicspath{ {images/} }
\everymath{\displaystyle}

% Professor/Course information
\title{Homework 2 - Part 2}
\author{Emily Dutile}
\date{February 2018}
\course{CS7295}{Info Viz}

\begin{document}


\section*{Stock Market Data Visualization}

\subsection*{Part A: Stock Market Visualizations}
In the historical prices visualization, we see an area plot of the Dow Jones Industrial Average. The stock market index shows the trading of a number of U.S. companies over a particular time period. The more advanced charting allows for better data analysis and comparison, and the table gives some explanation and insights. The area chart is interactive and the hover-over tooltip provides better analysis/exploration of a particular point.\\

In the Advanced Charting visualization, the user is able to select specific time frames and plotting the average line. The user selects a range, like the first visualization, along with a frequency and encoding type, along with numerous filters to allow the user to take different variables or events into consideration for a time period.

\subsection*{Part B: Stock Market Data}

\begin{center}
Table 1: Data Types for Dow Jones, NASDAQ, NYSE, and SP500
\end{center}
\begin{center} 
\begin{tabular}[h]{l l l l}
\textbf{Column} & \textbf{Data Type} \\
Date & quantitative \\
Open & quantitative \\ 
High & quantitative \\
Low & quantitative \\
Close & quantitative \\

\end{tabular}
\end{center}

\subsection*{Part C: Plotting with Python}
Please see iPython Notebook with markups explaining visualizations, decisions and encodings.

\end{document}
