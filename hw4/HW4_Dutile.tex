\documentclass{neu_handout}
\usepackage{url}
\usepackage{amssymb}
\usepackage{amsmath}
\usepackage{marvosym}
\usepackage{graphicx}
\usepackage[pdftex]{graphicx}
\usepackage{subfigure}
\graphicspath{ {images/} }
\everymath{\displaystyle}

% Professor/Course information
\title{Homework 4}
\author{Emily Dutile}
\date{March 2018}
\course{CS7295}{Info Viz}

\begin{document}


\section*{1 Dear Data}

In order to visualization my question, I decided to keep track of all of the things that I saw in a day that were beautiful to me. I thought I could gather some substantial data with the following categories: people, animals, plants, weather, objects, art, music, language, literature, and happiness. People, consisted of individuals like my boyfriend, friends at school and outside of school, and family. Although not necessarily weather, I did incorporate things such as "sunsets" in weather, and literature could also consist of articles or short stories rather than just novels. Every tally represented one of these moments. To discover where I might "see more beauty" during my day, I decided to keep track of the location of where I was at the time. Giving insight into what day it was also gave me an interesting idea to see if I think I see more beautiful things during the weekend or during my routinely day.\\

In order to think more abstractly and not fall back on a bar chart or some standard visualization, I thought it was best to group different types of beauty into some category was necessary to give some meaning to the data. I also thought that adding location and the day added another more granular level of detail and could add some interesting interpretation afterwards.

\section*{2 Journal Paper Review – Color Theory Review}

\textbf{Did you learn anything reading this article new not discussed in class?  What concept you found most interesting in the article?}

I did learn a few new things that were not discussed in class.\\\\

\textbf{In a paragraph (few sentences): Pretend you are talking to a friend who is completely unfamiliar with data visualization and color theory.  Summarize the most important take-away messages from this article and in-class for practicle advice on using and picking color for data visualizations and imaging.}

\section*{Boston Data Exploration}

Please see the iPython Notebook submitted in my NEU Google Drive (HW4-Dutile-BostonDataExploration.ipynb).

\section*{D3 basics}

1. https://jsfiddle.net/ \\

2. https://jsfiddle.net/



\end{document}
