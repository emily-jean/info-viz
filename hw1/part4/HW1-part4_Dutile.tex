\documentclass{neu_handout}
\usepackage{url}
\usepackage{amssymb}
\usepackage{amsmath}
\usepackage{marvosym}
\usepackage{graphicx}
\usepackage[pdftex]{graphicx}
\usepackage{subfigure}
\graphicspath{ {images/} }
\everymath{\displaystyle}

% Professor/Course information
\title{Homework 1 - Part 4}
\author{Emily Dutile}
\date{January 2018}
\course{CS7295}{Info Viz}

\begin{document}


\section*{Journal Paper Review}

\subsection*{Step 1}

In 2002, Chris Stolte, Diane Tang, and Pat Hanrahan presented Polaris, an interface for the exploration of multi-dimensional databases that extends the Pivot Table interface to directly generate a rich, expressive set of graphical displays that can be rapidly and incrementally developed. The goal was to give users a quick feedback loop during the start of the complex era of big data. \\

The authors introduce the features to Polaris that make it a novel idea and demonstrates its benefits against the limitations of other database exploration tools at that time. The features introduced that support the analysis process in large multi-dimensional databases meet the demands of data-dense displays, multiple display types, and exploratory interface. The paper addresses the complexities of integrating query and visualization schemas, and the design decision of using a formalism was innovative and necessary. Although the paper constructs a logical argument for the need of the exploratory process and Polaris's flexibility and ease through its visual interface, there are a few gaps. At the time, this solution did have limitations around features such as panning and zooming (coordinated data navigation). There is also a lack of support for hierarchical data. The features between tradition data warehouses and non-traditional business intelligence is a bit unclear as well..  \\

As stated in the homework assignment, this paper lead to the creation of Tableau. At the time, this was into the start of the complex big data era with massive tech companies and scientific projects creating huge corporate data warehouses. From this, the question was how do we extract meaning from all of this data? This was an absolute novel idea to fill the gap between the mismatch in design capabilities between tradition visualization and relational databases. Data visualization is a booming field, and these authors were fundamental to the contributions of this field. Information is not useful if a user cannot easily construct any meaning from it.

\subsection*{Step 2}
Use Figures 1 and 3 as a reference point to discuss the UI design and Figure 2
as a reference point to discuss the plotting logic and interaction options.\\

Polaris and Tableau software share many of the same fundamental ideas within the UI. Polaris's drag and drop interface, combining multiple data sources, grouping and sorting data, and calculating aggregations are all main features that are still in the software today. This cohesive architecture for coordinating visualization components is the bones of Tableau today. In both versions, by selecting a single mark in a graphic will display the values for the mark, and they utilize the same properties such as shape, size, orientation, and color. Quantitative variables are varying in one aspect of time, and ordinal/nominal mapping vs quantitative mapping exists.



\end{document}
